\section{Informazioni generali}
\subsection{Dettagli sull`incontro}
\begin{itemize}
    \item \textbf{Luogo}: Incontro in remoto su Discord
    \item \textbf{Data}: 04-11-2024
    \item \textbf{Ora di inizio}: 15:30
    \item \textbf{Ora di fine}: 16:00
    \item \textbf{Partecipanti}:
    \begin{itemize}
        \item Bergamin Elia
        \item Chilese Elena
        \item Diviesti Filippo
        \item Djossa Edgar
        \item Pincin Matteo 
        \item Soranzo Andrea  
    \end{itemize}
\end{itemize}

\section{Motivo della riunione}
In questo incontro abbiamo discusso i motivi che hanno causato la sospensione della valutazione della nostra candidatura per il capitolato scelto, analizzando dettagliatamente i punti da migliorare e le modifiche da apportarvi.
\section{Resoconto}

\subsection{Esposizione dei punti critici}
Grazie alle indicazioni riportate nel documento \textit{“Avanzamento Revisioni.xlsx”} ed in seguito alla lezione tenutasi questa mattina, abbiamo compreso cosa c'è da migliorare e ciascun membro del gruppo ha espresso le proprie idee sul da farsi. \\
Sono quindi emerse nuove azioni da intraprendere per migliorare la candidatura e, in generale, per migliorare la gestione della documentazione.

\subsection{Discussione su cosa sviluppare}
A seguito della discussione abbiamo deciso di sviluppare un’interfaccia pubblica di accesso alla documentazione, tramite \textit{GitHub Pages}, aggiornata automaticamente utilizzando le \textit{GitHub Actions}, creeremo inoltre dei branch per lo sviluppo della documentazione, i file LaTex verranno compilati solo al momento del merging tramite pull request nel main branch, previa approvazione.\\
È emersa la necessità di modificare il documento riguardante la \textit{Dichiarazione degli impegni}, dopo che ci è stata fatta notare la mancanza della motivazione alla base della rotazione dei ruoli, e andremo ad aggiungere il versionamento nel documento in questione, in quanto assente.

\section{Prossimi obiettivi}
   \begin{itemize}
        \item Apportare le modifiche sopra indicate
        \item Ripresentare la candidatura
    \end{itemize}