% Insert content here
\section{Informazioni generali}
\subsection{Dettagli sull'incontro}
\begin{itemize}
    \item \textbf{Luogo}: Discord
    \item \textbf{Data}: 08-11-2024
    \item \textbf{Ora di inizio}: 15:00
    \item \textbf{Ora di fine}: 16:30
    \item \textbf{Partecipanti}: 
    \begin{itemize}
        \item Bergamin Elia
        \item Chilese Elena
        \item Diviesti Filippo
        \item Djossa Edgar
        \item Pincin Matteo 
        \item Soranzo Andrea  
    \end{itemize}
\end{itemize}

\section{Motivo della riunione}
In questo incontro abbiamo discusso, definito e evoluto alcuni principi del Way of Working, sistemato alcuni template per la creazione di documenti,
cominciato a pianificare le milestones a lungo termine e creato il powerpoint per il secondo diario di bordo.

\section{Resoconto}
\subsection{Retrospettiva}
Inizialmente ci siamo concentrati sull`analisi di attività passate e i nostri errori commessi, sopratutto quelli evidenziati dal Prof. Vardanega Tullio durante la presentazione
delle candidature. Da questo è stato deciso di cambiare la struttura del changelog aggiungendo una colonna "Verificatore" nella quale andrà inserito il nome del verificatore di quella versione di documento.

\subsection{Aggiornamento del Way of Working}
\subsubsection{Versionamento Documentazione}
Dopo il consiglio del Prof. Vardanega Tullio circa l`automatizzaione della creazione di documenti e relativa page dove andare a mostrarli abbiamo ragionato 
sul workflow da adottare per gestire il versionamento e cercare di lavorare in modo più asincrono possibile.
In breve il workflow deciso:
\begin{enumerate}
    \item Scaricarsi in locale il template
    \item Aprire un primario branch per quel documento
    \item Per ogni modifica aprire un sotto branch derivato dal primaio branch creato per il determinato documento
    \item Compilare/modificare il documento
    \item Una volta terminata la modifica aprire una pullrequest verso il branch primario
    \item Successivamente un verificatore dovrà commentare eventuali modifiche da apportare al documento oppure accettarla nel caso vada bene
    \item Una volta terminato il documento creare una pullrequest verso il main per approvarlo e pubblicare il tutto sulla page
\end{enumerate}
Tutto verrà poi specificato nel documento \textit{Norme di Progetto} che è in corso di redazione.
\subsubsection{Backlog}
Oltre al versionamento abbiamo deciso di introdurre un meccanismo di tracciamento e backlog spostandoci da un normale documento di google docs all board kanban di github legata
alle issue che mano a mano verranno create.


\subsection{Pianificazione}
Abbiamo cominciato in oltre a pianificare milesones a lungo termine, in specifico la RTB dandoci una data massima da rispettare ovvero il 17/01/2025. Successivamente negli ultimi minuti della riunione
ci siamo focalizzati sulla creazione di template per i documenti di \textit{Piano di progetto} e \textit{Norme di Progetto}.

\section{Prossimi obiettivi}
   \begin{itemize}
        \item Iniziare analisi dei requisiti
        \item Proseguire la redazione dei documenti \textit{Piano di progetto} e \textit{Norme di Progetto}
        \item Avvisare il proponente, definire canali di comunicazione e incontri futuri
    \end{itemize}